\chapter{Sobre la concordia y la discordia}

Una Operación militar significa un gran esfuerzo para el pueblo, y la guerra puede durar muchos años para obtener una victoria de un día. Así pues, fallar en conocer la situación de los adversarios por economizar en aprobar gastos para investigar y estudiar a la oposición es extremadamente inhumano, y no es típico de un buen jefe militar, de un consejero de gobierno, ni de un gobernante victorioso. Por lo tanto, lo que posibilita a un gobierno inteligente y a un mando militar sabio vencer a los demás y lograr triunfos extraordinarios con esa información esencial.

La información previa no puede obtenerse de fantasmas ni espíritus, ni se puede tener por analogía, ni descubrir mediante cálculos. Debe obtenerse de personas; personas que conozcan la situación del adversario.

Existen cinco clases de espías: el espía nativo, el espía interno, el doble agente, el espía liquidable, y el espía flotante. Cuando están activos todos ellos, nadie conoce sus rutas: a esto se le llama genio organizativo, y se aplica al gobernante.

Los espías nativos se contratan entre los habitantes de una localidad. Los espías internos se contratan entre los funcionarios enemigos. Los agentes dobles se contratan entre los espías enemigos. Los espías liquidables transmiten falsos datos a los espías enemigos. Los espías flotantes vuelven para traer sus informes.

Entre los funcionarios del régimen enemigo, se hallan aquéllos con los que se puede establecer contacto y a los que se puede sobornar para averiguar la situación de su país y descubrir cualquier plan que se trame contra ti, también pueden ser utilizados para crear desavenencias y desarmonía.

En consecuencia, nadie en las fuerzas armadas es tratado con tanta familiaridad como los espías, ni a nadie se le otorgan recompensas tan grandes como a ellos, ni hay asunto más secreto que el espionaje.

Si no se trata bien a los espías, pueden convertirse en renegados y trabajar para el enemigo.

No se pueden utilizar a los espías sin sagacidad y conocimiento; no puede uno servirse de espías sin humanidad y justicia, no se puede obtener la verdad de los espías sin sutileza. Ciertamente, es un asunto muy delicado. Los espías son útiles en todas partes.

Cada asunto requiere un conocimiento previo.

Si algún asunto de espionaje es divulgado antes de que el espía haya informado, éste y el que lo haya divulgado deben eliminarse.

Siempre que quieras atacar a un ejército, asediar una ciudad o atacar a una persona, has de conocer previamente la identidad de los generales que la defienden, de sus aliados, sus visitantes, sus centinelas y de sus criados; así pues, haz que tus espías averigüen todo sobre ellos.

Siempre que vayas a atacar y a combatir, debes conocer primero los talentos de los servidores del enemigo, y así puedes enfrentarte a ellos según sus capacidades.

Debes buscar a agentes enemigos que hayan venido a espiarte, sobornarlos e inducirlos a pasarse a tu lado, para poder utilizarlos como agentes dobles. Con la información obtenida de esta manera, puedes encontrar espías nativos y espías internos para contratarlos. Con la información obtenida de éstos, puedes fabricar información falsa sirviéndote de espías liquidables. Con la información así obtenida, puedes hacer que los espías flotantes actúen según los planes previstos.

Es esencial para un gobernante conocer las cinco clases de espionaje, y este conocimiento depende de los agentes dobles; así pues, éstos deben ser bien tratados.

Así, sólo un gobernante brillante o un general sabio que pueda utilizar a los más inteligentes para el espionaje puede estar seguro de la victoria. El espionaje es esencial para las operaciones militares, y los ejércitos dependen de él para llevar a cabo sus acciones.

No será ventajoso para el ejército actuar sin conocer la situación del enemigo, y conocer la situación del enemigo no es posible sin el espionaje.

\centering
\vspace{11mm}
\Huge\textbf{FIN}
