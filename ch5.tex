\chapter{Sobre la firmeza}

La fuerza es la energía acumulada o la que se percibe. Esto es muy cambiante. Los expertos son capaces de vencer al enemigo creando una percepción favorable en ellos, así obtener la victoria sin necesidad de ejercer su fuerza.

Gobernar sobre muchas personas como si fueran poco es una cuestión de dividir las en grupos o sectores: es organización. Batallar contra un gran número de tropas como si fueran pocas es una cuestión de demostrar la fuerza, símbolos y señales.

Se refiere a lograr una percepción de fuerza y poder en la oposición. En el campo de batalla se refiere a las formaciones y banderas utilizadas para desplegar las tropas y coordinar sus movimientos.

Lograr que el ejército sea capaz de combatir contra el adversario sin ser derrotado es una cuestión de emplear métodos ortodoxos o heterodoxos.

La ortodoxia y la heterodoxia no es algo fijo, sino que se utilizan como un ciclo. Un emperador que fue un famoso guerrero y administrador hablaba de manipular las percepciones de los adversarios sobre lo que es ortodoxo y heterodoxo, y después atacar inesperadamente, combinando ambos métodos hasta convertirlo en uno, volviéndose así indefinible para el enemigo.

Que el efecto de las fuerzas sea como el de piedras arrojadas sobre huevos, es una cuestión de lleno y vacío.

Cuando induces a los adversarios a atacarte en tu territorio, su fuerza siempre está vacía (en desventaja); mientras que no compitas en lo que son los mejores, tu fuerza siempre estará llena. Atacar con lo vacío contra lo lleno es como arrojar piedras sobre huevos: de seguro se rompen.

Cuando se entabla una batalla de manera directa, la victoria se gana por sorpresa. El ataque directo es ortodoxo. El ataque indirecto es heterodoxo.

Sólo hay dos clases de ataques en la batalla: el extraordinario por sorpresa y el directo ordinario, pero sus variantes son innumerables. Lo ortodoxo y lo heterodoxo se originan recíprocamente, como un círculo sin comienzo ni fin; ¿quién podría agotarlos?

Cuando la velocidad del agua que fluye alcanza el punto en el que puede mover las piedras, ésta es la fuerza directa. Cuando la velocidad y maniobrabilidad del halcón es tal que puede atacar y matar, esto es precisión. Lo mismo ocurre con los guerreros expertos: su fuerza es rápida, su precisión certera. Su fuerza es como disparar una catapulta, su precisión es dar en el objetivo previsto y causar el efecto esperado.

El desorden llega del orden, la cobardía surge del valor, la debilidad brota de la fuerza.

Si quieres fingir desorden para convencer a tus adversarios y distraerlos, primero tienes que organizar el orden, porque sólo entonces puedes crear un desorden artificial. Si quieres fingir cobardía para conocer la estrategia de los adversarios, primero tienes que ser extremadamente valiente, porque sólo entonces puedes actuar como tímido de manera artificial. Si quieres fingir debilidad para inducir la arrogancia en tus enemigos, primero has de ser extremadamente fuerte porque sólo entonces puedes pretender ser débil.

El orden y el desorden son una cuestión de organización; la cobardía es una cuestión valentía y la de ímpetu; la fuerza y la debilidad son una cuestión de la formación en la batalla.

Cuando un ejército tiene la fuerza del ímpetu (percepción), incluso el tímido se vuelve valiente, cuando pierde la fuerza del ímpetu, incluso el valiente se convierte en tímido. Nada está fijado en las leyes de la guerra: éstas se desarrollan sobre la base del ímpetu.

Con astucia se puede anticipar y lograr que los adversarios se convenzan a sí mismos cómo proceder y moverse; les ayuda a caminar por el camino que les traza. Hace moverse a los enemigos con la perspectiva del triunfo, para que caigan en la emboscada.

Los buenos guerreros buscan la efectividad en la batalla a partir de la fuerza del ímpetu (percepción) y no dependen sólo de la fuerza de sus soldados. Son capaces de escoger a la mejor gente, desplegarlos adecuadamente y dejar que la fuerza del ímpetu logre sus objetivos.

Cuando hay entusiasmo, convicción, orden, organización, recursos, compromiso de los soldados, tienes la fuerza del ímpetu, y el tímido es valeroso. Así es posible asignar a los soldados por sus capacidades, habilidades y encomendarle deberes y responsabilidades adecuadas. El valiente puede luchar, el cuidadoso puede hacer de centinela, y el inteligente puede estudiar, analizar y comunicar. Cada cual es útil.

Hacer que los soldados luchen permitiendo que la fuerza del ímpetu haga su trabajo es como hacer rodar rocas. Las rocas permanecen inmóviles cuando están en un lugar plano, pero ruedan en un plano inclinado; se quedan fijas cuando son cuadradas, pero giran si son redondas. Por lo tanto, cuando se conduce a los hombres a la batalla con astucia, el impulso es como rocas redondas que se precipitan montaña abajo: ésta es la fuerza que produce la victoria. 
