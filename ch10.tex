\chapter{Sobre la topología}

Algunos terrenos son fáciles, otros difíciles, algunos neutros, otros estrechos, accidentados o abiertos.

Cuando el terreno sea accesible, sé el primero en establecer tu posición, eligiendo las alturas soleadas; una posición que sea adecuada para transportar los suministros; así tendrás ventaja cuando libres la batalla.

Cuando estés en un terreno difícil de salir, estás limitado. En este terreno, si tu enemigo no está preparado, puedes vencer si sigues adelante, pero si el enemigo está preparado y sigues adelante, tendrás muchas dificultades para volver de nuevo a él, lo cual jugará en contra tuya.

Cuando es un terreno desfavorable para ambos bandos, se dice que es un terreno neutro. En un terreno neutro, incluso si el adversario te ofrece una ventaja, no te aproveches de ella: retírate, induciendo a salir a la mitad de las tropas enemigas, y entonces cae sobre él aprovechándote de esta condición favorable.

En un terreno estrecho, si eres el primero en llegar, debes ocuparlo totalmente y esperar al adversario. Si él llega antes, no lo persigas si bloquea los desfiladeros. Persíguelo sólo si no los bloquea.

En terreno accidentado, si eres el primero en llegar, debes ocupar sus puntos altos y soleados y esperar al adversario. Si éste los ha ocupado antes, retírate y no lo persigas.

En un terreno abierto, la fuerza del ímpetu se encuentra igualada, y es difícil provocarle a combatir de manera desventajosa para él.

Entender estas seis clases de terreno es la responsabilidad principal del general, y es imprescindible considerarlos.

Éstas son las configuraciones del terreno; los generales que las ignoran salen derrotados.

Así pues, entre las tropas están las que huyen, la que se retraen, las que se derrumban, las que se rebelan y las que son derrotadas. Ninguna de estas circunstancias constituye desastres naturales, sino que son debidas a los errores de los generales.

Las tropas que tienen el mismo ímpetu, pero que atacan en proporción de uno contra diez, salen derrotadas. Los que tienen tropas fuertes pero cuyos oficiales son débiles, quedan retraídos.

Los que tienen soldados débiles al mando de oficiales fuertes, se verán en apuros. Cuando los oficiales superiores están enco lerizados y son violentos, y se enfrentan al enemigo por su cuenta y por despecho, y cuando los generales ignoran sus capacidades, el ejército se desmoronará.

Como norma general, para poder vencer al enemigo, todo el mando militar debe tener una sola intención y todas las fuerzas militares deben cooperar.

Cuando los generales son débiles y carecen de autoridad, cuando las órdenes no son claras, cuando oficiales y soldados no tienen solidez y las formaciones son anárquicas, se produce revuelta.

Los generales que son derrotados son aquellos que son incapaces de calibrar a los adversarios, entran en combate con fuerzas superiores en número o mejor equipadas, y no seleccionan a sus tropas según los niveles de preparación de las mismas.

Si empleas soldados sin seleccionar a los preparados de los no preparados, a los arrojados y a los timoratos, te estás buscando tu propia derrota.

Estas son las seis maneras de ser derrotado. La comprensión de estas situaciones es la responsabilidad suprema de los generales y deben ser consideradas.

La primera es no calibrar el número de fuerzas; la segunda, la ausencia de un sistema claro de recompensas y castigos; la tercera, la insuficiencia de entrenamiento; la cuarta es la pasión irracional; la quinta es la ineficacia de la ley del orden; y la sexta es el fallo de no seleccionar a los soldados fuertes y resueltos.

La configuración del terreno puede ser un apoyo para el ejército; para los jefes militares, el curso de la acción adecuada es calibrar al adversario para asegurar la victoria y calcular los riesgos y las distancias. Salen vencedores los que libran batallas conociendo estos elementos; salen derrotados los que luchan ignorándolos.

Por lo tanto, cuando las leyes de la guerra señalan una victoria segura es claramente apropiado entablar batalla, incluso si el gobierno ha dada órdenes de no atacar. Si las leyes de la guerra no indican una victoria segura, es adecuado no entrar en batalla, aunque el gobierno haya dada la orden de atacar. De este modo se avanza sin pretender la gloria, se ordena la retirada sin evitar la responsabilidad, con el único propósito de proteger a la población y en beneficio también del gobierno; así se rinde un servicio valioso a la nación.

Avanzar y retirarse en contra de las órdenes del gobierno no se hace por interés personal, sino para salvaguardar las vidas de la población y en auténtico beneficio del gobierno. Servidores de esta talla son muy útiles para un pueblo.

Mira por tus soldados como miras por un recién nacido; así estarán dispuestos a seguirte hasta los valles más profundos; cuida de tus soldados como cuidas de tus queridos hijos, y morirán gustosamente contigo.

Pero si eres tan amable con ellos que no los puedes utilizar, si eres tan indulgente que no les puedes dar órdenes, tan informal que no puedes disciplinarlos, tus soldados serán como niños mimados y, por lo tanto, inservibles.

Las recompensas no deben utilizarse solas, ni debe confiarse solamente en los castigos. En caso contrario, las tropas, como niños mimosos, se acostumbran a disfrutar o a quedar resentidas por todo. Esto es dañino y los vuelve inservibles.

Si sabes que tus soldados son capaces de atacar, pero ignoras si el enemigo es invulnerable a un ataque, tienes sólo la mitad de posibilidades de ganar. Si sabes que tu enemigo es vulnerable a un ataque, pero ignoras si tus soldados son capaces de atacar, sólo tienes la mitad de posibilidades de ganar. Si sabes que el enemigo es vulnerable a un ataque, y tus soldados pueden llevarlo a cabo, pero ignoras si la condición del terreno es favorable para la batalla, tienes la mitad de probabilidades de vencer.

Por lo tanto, los que conocen las artes marciales no pierden el tiempo cuando efectúan sus movimientos, ni se agotan cuando atacan. Debido a esto se dice que cuando te conoces a ti mismo y conoces a los demás, la victoria no es un peligro; cuando conoces el cielo y la tierra, la victoria es inagotable.
