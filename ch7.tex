\chapter{Sobre el enfrentamiento directo e indirecto}

La regla ordinaria para el uso del ejército es que el mando del ejército reciba órdenes de las autoridades civiles y después reúne y concentra a las tropas, acuartelándolas juntas. Nada es más difícil que la lucha armada.

Luchar con otros cara a cara para conseguir ventajas es lo más arduo del mundo.

La dificultad de la lucha armada es hacer cercanas las distancias largas y convertir los problemas en ventajas.

Mientras que das la apariencia de estar muy lejos, empiezas tu camino y llegas antes que el enemigo.

Por lo tanto, haces que su ruta sea larga, atrayéndole con la esperanza de ganar. Cuando emprendes la marcha después que los otros y llegas antes que ellos, conoces la estrategia de hacer que las distancias sean cercanas.

Sírvete de una unidad especial para engañar al enemigo atrayéndole a una falsa persecución, haciéndole creer que el grueso de tus fuerzas está muy lejos; entonces, lanzas una fuerza de ataque sorpresa que llega antes, aunque emprendió el camino después.

Por consiguiente, la lucha armada puede ser provechosa y puede ser peligrosa. Para el experto es provechosa, para el inexperto peligrosa.

Movilizar a todo el ejército para el combate en aras de obtener alguna ventaja tomaría mucho tiempo, pero combatir por una ventaja con un ejército incompleto tendría como resultado una falta de recursos.

Si te movilizas rápidamente y sin parar día y noche, recorriendo el doble de la distancia habitual, y si luchas por obtener alguna ventaja a miles de kilómetros, tus jefes militares serán hechos prisioneros. Los soldados que sean fuertes llegarán allí primero, los más cansados llegarán después - como regla ge neral, sólo lo conseguirá uno de cada diez.

Cuando la ruta es larga las tropas se cansan; si han gastado su fuerza en la movilización, llegan agotadas mientras que sus adversarios están frescos; así pues, es seguro que serán atacadas.

Combatir por una vent aja a cincuenta kilómetros de distancia frustrará los planes del mando, y, como regla general, sólo la mitad de los soldados lo harán.

Si se combate por obtener una ventaja a treinta kilómetros de distancia, sólo dos de cada tres soldados los recorrerán.

Así pues, un ejército perece si no está equipado, si no tiene provisiones o si no tiene dinero.

Estas tres cosas son necesarias: no puedes combatir para ganar con un ejército no equipado, o sin provisiones, lo que el dinero facilita.

Por tanto, si ignoras los planes de tus rivales, no puedes hacer alianzas precisas.

A menos que conozcas las montañas y los bosques, los desfiladeros y los pasos, y la condición de los pantanos, no puedes maniobrar con una fuerza armada. A menos que utilices guías locales, no puedes aprovecharte de las ventajas del terreno.

Sólo cuando conoces cada detalle de la condición del terreno puedes maniobrar y guerrear.

Por consiguiente, una fuerza militar se usa según la estrategia prevista, se moviliza mediante la esperanza de recompensa, y se adapta mediante la división y la combinación.

Una fuerza militar se establece mediante la estrategia en el sentido de que distraes al enemigo para que no pueda conocer cuál es tu situación real y no pueda imponer su supremacía. Se moviliza mediante la esperanza de recompensa, en el sentido de que entra en acción cuando ve la posibilidad de obtener una ventaja. Dividir y volver a hacer combinaciones de tropas se hace para confundir al adversario y observar cómo reacciona frente a ti; de esta manera puedes adaptarte para obtener la victoria.

Por eso, cuando una fuerza militar se mueve con rapidez es como el viento; cuando va lentamente es como el bosque; es voraz como el fuego e inmóvil como las montañas.

Es rápida como el viento en el sentido que llega sin avisar y desaparece como el relámpago. Es como un bosque porque tiene un orden. Es voraz como el fuego que devasta una planicie sin dejar tras sí ni una brizna de hierba. Es inmóvil como una montaña cuando se acuartela.

Es tan difícil de conocer como la oscuridad; su movimiento es como un trueno que retumba.

Para ocupar un lugar, divide a tus tropas. Para expandir tu territorio, divide los beneficios.

La regla general de las operaciones militares es desproveer de alimentos al enemigo todo lo que se pueda. Sin embargo, en localidades donde la gente no tiene mucho, es necesario dividir a las tropas en grupos más pequeños para que puedan tomar en diversas partes lo que necesitan, ya que sólo así tendrán suficiente.

En cuanto a dividir el botín, significa que es necesario repartirlo entre las tropas para guardar lo que ha sido ganado, no dejando que el enemigo lo recupere.

Actúa después de haber hecho una estimación. Gana el que conoce primero la medida de lo que está lejos y lo que está cerca: ésta es la regla general de la lucha armada.

El primero que hace el movimiento es el ``invitado'', el último es el ``anfitrión''. El ``invitado'' lo tiene difícil, el ``anfitrión lo tiene fácil''. Cerca y lejos significan desplazamiento: el cansancio, el hambre y el frío sur gen del desplazamiento.

Un antiguo libro que trata de asuntos militares dice: ``Las palabras no son escuchadas, par aeso se hacen los símbolos y los tambores. Las banderas y los estandartes se hacen a causa de la ausencia de visibilidad''. Símbolos, tambores, banderas y estandartes se utilizan para concentrar y unificar los oídos y los ojos de los soldados. Una vez que están unificados, el valiente no puede actuar solo, ni el tímido puede retirarse solo: ésta es la regla general del empleo de un grupo.

Unificar los oídos y los ojos de los soldados significa hacer que miren y escuchen al unísono de manera que no caigan en la confusión y el desorden. La señales se utilizan para indicar direcciones e impedir que los individuos vayan a donde se les antoje.

Así pues, en batallas nocturnas, utiliza fuegos y tambores, y en batallas diurnas sírvete de banderas y estandartes, para manipular los oídos y los ojos de los soldados.

Utiliza muchas señales para confundir las percepciones del enemigo y hacerle temer tu temible poder militar.

De esta forma, haces desaparecer la energía de sus ejércitos y desmoralizas a sus generales.

En primer lugar, has de ser capaz de mantenerte firme en tu propio corazón; sólo entonces puedes desmoralizar a los generales enemigos. Por esto, la tradición afirma que los habitantes de otros tiempos tenían la firmeza para desmoralizar, y la antigua ley de los que conducían carros de combate decía que cuando la mente original es firme, la energía fresca es victoriosa.

De este modo, la energía de la mañana está llena de ardor, la del mediodía decae y la energía de la noche se retira; en consecuencia, los expertos en el manejo de las armas prefieren la energía entusiasta, atacan la decadente y la que se bate en retirada. Son ellos los que dominan la energía.

Cualquier débil en el mundo se dispone a combatir en un minuto si se siente animado, pero cuando se trata realmente de tomar las armas y de entrar en batalla, es poseído por la energía; cuando esta energía se desvanece, se detendrá, estará asustado y se arrepentirá de haber comenzado. La razón por la que esa clase de ejércitos miran por encima del hombro a enemigos fuertes, lo mismo que miran a las doncellas vírgenes, es porque se están aprovechando de su agresividad, estimulada por cualquier causa.

Utilizar el orden para enfrentarse al desorden, utilizar la calma para enfrentarse con los que se agitan, esto es dominar el corazón.

A menos que tu corazón esté totalmente abierto y tu mente en orden, no puedes esperar ser capaz de adaptarte a responder sin límites, a manejar los acontecimientos de manera infalible, a enfrentarte a dificultades graves e inesperadas sin turbarte, dirigiendo cada cosa sin confusión.

Dominar la fuerza es esperar a los que vienen de lejos, aguardar con toda comodidad a los que se han fatigado, y con el estómago saciado a los hambrientos.

Esto es lo que se quiere decir cuando se habla de atraer a otros hacia donde estás, al tiempo que evitas ser inducido a ir hacia donde están ellos.

Evitar la confrontación contra formaciones de combate bien ordenadas y no atacar grandes batallones constituye el dominio de la adaptación.

Por tanto, la regla general de las operaciones militares es no enfrentarse a una gran montaña ni oponerse al enemigo de espaldas a ésta.

Esto significa que si los adversarios están en un terreno elevado, no debes atacarles cuesta arriba, y que cuando efectúan una carga cuesta abajo, no debes hacerles frente.

No persigas a los enemigos cuando finjan una retirada, ni ataques tropas expertas.

Si los adversarios huyen de repente antes de agotar su energía, seguramente hay emboscadas esperándote para atacar a tus tropas; en este caso, debes retener a tus oficiales para que no se lancen en su persecución.

No consumas la comida de sus soldados.

Si el enemigo abandona de repente sus provisiones, éstas han de ser probadas antes de ser comidas, por si están envenenadas.

No detengas a ningún ejército que esté en camino a su país.

Bajo estas circunstancias, un adversario luchará hasta la muerte. Hay que dejarle una salida a un ejército rodeado.

Muéstrales una manera de salvar la vida para que no estén dispuestos a luchar hasta la muerte, y así podrás aprovecharte para atacarles.

No presiones a un enemigo desesperado.

Un animal agotado seguirá luchando, pues esa es la ley de la naturaleza. Estas son las leyes de las operaciones militares.
