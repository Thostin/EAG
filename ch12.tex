\chapter{Sobre el arte de atacar por el fuego}

Existen cinco clases de ataques mediante el fuego: quemar a las personas, quemar los suministros, quemar el equipo, quemar los almacenes y quemar las armas.

El uso del fuego tiene que tener una base, y exige ciertos medios. Existen momentos adecuados para encender fuegos, concretamente cuando el tiempo es seco y ventoso.

Normalmente, en ataques mediante el fuego es imprescindible seguir los cambios producidos por éste. Cuando el fuego está dentro del campamento enemigo, prepárate rápidamente desde fuera. Si los soldados se mantienen en calma cuando el fuego se ha declarado, espera y no ataques. Cuando el fuego alcance su punto álgido, síguelo, si puedes; si no, espera.

En general, el fuego se utiliza para sembrar la confusión en el enemigo y así poder atacarle.

Cuando el fuego puede ser prendido en campo abierto, no esperes a hacerlo en su interior; hazlo cuando sea oportuno.

Cuando el fuego sea atizado par el viento, no ataques en dirección contraria a éste.

No es eficaz luchar contra el ímpetu del fuego, porque el enemigo luchará en este caso hasta la muerte.

Si ha soplado el viento durante el día, a la noche amainará.

Un viento diurno cesará al anochecer; un viento nocturno cesará al amanecer.

Los ejércitos han de saber que existen variantes de las cinco clases de ataques mediante el fuego, y adaptarse a éstas de manera racional.

No basta saber cómo atacar a los demás con el fuego, es necesario saber cómo impedir que los demás te ataquen a ti.

Así pues, la utilización del fuego para apoyar un ataque significa claridad, y la utilización del agua para apoyar un ataque significa fuerza. El agua puede incomunicar, pero no puede arrasar.

El agua puede utilizarse para dividir a un ejército enemigo, de manera que su fuerza se desuna y la tuya se fortalezca.

Ganar combatiendo o llevar a cabo un asedio victorioso sin recompensar a los que han hecho méritos trae mala fortuna y se hace merecedor de ser llamado avaro. Por eso se dice que un gobierno esclarecido lo tiene en cuenta y que un buen mando militar recompensa el mérito. No moviliza a sus tropas cuando no hay ventajas que obtener, ni actúa cuando no hay nada que ganar, ni luchan cuando no existe peligro.

Las armas son instrumentos de mal augurio, y la guerra es un asunto peligroso. Es indispensable impedir una derrota desastrosa, y por lo tanto, no vale la pena movilizar un ejército por razones insignificantes: Las armas sólo deben utilizarse cuando no existe otro remedio.

Un gobierno no debe movilizar un ejército por ira, y los jefes militares no deben provocar la guerra por cólera.

Actúa cuando sea beneficioso; en caso contrario, desiste. La ira puede convertirse en alegría, y la cólera puede convertirse en placer, pero un pueblo destruido no puede hacérsele renacer, y la muerte no puede convertirse en vida. En consecuencia, un gobierno esclarecido presta atención a todo esto, y un buen mando militar lo tiene en cuenta. Ésta es la manera de mantener a la nación a salvo y de conservar intacto a su ejército.

