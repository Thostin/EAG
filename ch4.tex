\chapter{Sobre la medida en la disposición de los medios}

Antiguamente, los guerreros expertos se hacían a sí mismos invencibles en primer lugar, y después aguardaban para descubrir la vulnerabilidad de sus adversarios.

Hacerte invencible significa conocerte a ti mismo; aguardar para descubrir la vulnerabilidad del adversario significa conocer a los demás.

La invencibilidad está en uno mismo, la vulnerabilidad en el adversario.

Por esto, los guerreros expertos pueden ser invencibles, pero no pueden hacer que sus adversarios sean vulnerables.

Si los adversarios no tienen orden de batalla sobre el que informarse, ni negligencias o fallos de los que aprovecharse, ¿cómo puedes vencerlos, aunque estén bien pertrechados? Por esto es por lo que se dice que la victoria puede ser percibida, pero no fabricada.

La invencibilidad es una cuestión de defensa, la vulnerabilidad, una cuestión de ataque.

Mientras no hayas observado vulnerabilidades en el orden de batalla de los adversarios, oculta tu propia formación de ataque, y prepárate para ser invencible, con la finalidad de preservarte. Cuando los adversarios tienen órdenes de batalla vulnerables, es el momento de salir aatacarlos.

La defensa es para tiempos de escasez, el ataque para tiempos de abundancia.

Los expertos en defensa se esconden en las profundidades de la tierra; los expertos en maniobras de ataque se esconden en las más elevadas alturas del cielo. De esta manera pueden protegerse y lograr la victoria total.

En situaciones de defensa, acalláis las voces y borráis las huellas, escondidos como fantasmas y espíritus bajo tierra, invisibles para todo el mundo. En situaciones de ataque, vuestro movimiento es rápido y vuestro grito fulgurante, veloz como el trueno y el relámpago, para los que no se puede uno preparar, aunque vengan del cielo.

Prever la victoria cuando cualquiera la puede conocer no constituye verdadera destreza. Todo el mundo elogia la victoria ganada en batalla, pero esa victoria no es realmente tan buena.

Todo el mundo elogia la victoria en la batalla, pero lo verdaderamente deseable es poder ver el mundo de lo sutil y darte cuenta del mundo de lo oculto, hasta el punto de ser capaz de alcanzar la victoria donde no existe forma.

No se requiere mucha fuerza para levantar un cabello, no es necesario tener una vista aguda para ver el sol y la luna, ni se necesita tener mucho oído para escuchar el retumbar del trueno.

Lo que todo el mundo conoce no se llama sabiduría; la victoria sobre los demás obtenida por medio de la batalla no se considera una buena victoria.

En la antigüedad, los que eran conocidos como buenos guerreros vencían cuando era fácil vencer.

Si sólo eres capaz de asegurar la victoria tras enfrentarte a un adversario en un conflicto armado, esa victoria es una dura victoria. Si eres capaz de ver lo sutil y de darte cuenta de lo oculto, irrumpiendo antes del orden de batalla, la victoria así obtenida es una victoria fácil.

En consecuencia, las victorias de los buenos guerreros no destacan por su inteligencia o su bravura. Así pues, las victorias que ganan en batalla no son debidas a la suerte. Sus victorias no son casualidades, sino que son debidas a haberse situado previamente en posición de poder ganar con seguridad, imponiéndose sobre los que ya han perdido de antemano.

La gran sabiduría no es algo obvio, el mérito grande no se anuncia. Cuando eres capaz de ver lo sutil, es fácil ganar; ¿qué tiene esto que ver con la inteligencia o la bravura?

Cuando se resuelven los problemas antes de que surjan, ¿quién llama a esto inteligencia? Cuando hay victoria sin batalla, ¿quién habla de bravura?

Así pues, los buenos guerreros toman posición en un terreno en el que no pueden perder, y no pasan por alto las condiciones que hacen a su adversario proclive a la derrota.

En consecuencia, un ejército victorioso gana primero y entabla la batalla después; un ejército derrotado lucha primero e intenta obtener la victoria después.

Esta es la diferencia entre los que tienen estrategia y los que no tienen planes premeditados.

Los que utilizan bien las armas cultivan el Camino y observan las leyes. Así pueden gobernar prevaleciendo sobre los corruptos.

Servirse de la armonía para desvanecer la oposición, no atacar un ejército inocente, no hacer prisioneros o tomar botín par donde pasa el ejército, no cortar los árboles ni contaminar los pozos, limpiar y purificar los templos de las ciudades y montañas del camino que atraviesas, no repetir los errores de una civilización decadente, a todo esto, se llama el Camino y sus leyes.

Cuando el ejército está estrictamente disciplinado, hasta el punto en que los soldados morirían antes que desobedecer las órdenes, y las recompensas y los castigos merecen confianza y están bien establecidos, cuando los jefes y oficiales son capaces de actuar de esta forma, pueden vencer a un Príncipe enemigo corrupto.

Las reglas militares son cinco: medición, valoración, cálculo, comparación y victoria. El terreno da lugar a las mediciones, éstas dan lugar a las valoraciones, las valoraciones a los cálculos, éstos a las comparaciones, y las comparaciones dan lugar a las victorias.

Mediante las comparaciones de las dimensiones puedes conocer dónde se haya la victoria o la derrota.

En consecuencia, un ejército victorioso es como un kilo comparado con un gramo; un ejército derrotado es como un gramo comparado con un kilo.

Cuando el que gana consigue que su pueblo vaya a la batalla como si estuviera dirigiendo una gran corriente de agua hacia un cañón profundo, esto es una cuestión de orden de batalla.
 
Cuando el agua se acumula en un cañón profundo, nadie puede medir su cantidad, lo mismo que nuestra defensa no muestra su forma. Cuando se suelta el agua, se precipita hacia abajo como un torrente, de manera tan irresistible como nuestro propio ataque.
 
