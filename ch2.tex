\chapter{Sobre la iniciación de las acciones}

Una vez comenzada la batalla, aunque estés ganando, de continuar por mucho tiempo, desanimará a tus tropas y embotará tu espada. Si estás sitiando una ciudad, agotarás tus fuerzas. Si mantienes a tu ejército durante mucho tiempo en campaña, tus suministros se agotarán.

Las armas son instrumentos de mala suerte; emplearlas por mucho tiempo producirá calamidades. Como se ha dicho: ``Los que a hierro matan, a hierro mueren''. Cuando tus tropas están desanimadas, tu espada embotada, agotadas tus fuerzas y tus suministros son escasos, hasta los tuyos se aprovecharán de tu debilidad para sublevarse. Entonces, aunque tengas consejeros sabios, al final no podrás hacer que las cosas salgan bien.

Por esta causa, he oído hablar de operaciones militares que han sido torpes y repentinas, pero nunca he visto a ningún experto en el arte de la guerra que mantuviese la campaña por mucho tiempo. Nunca es beneficioso para un país dejar que una operación militar se prolongue por mucho tiempo.

Como se dice comúnmente, sé rápido como el trueno que retumba antes de que hayas podido taparte los oídos, veloz como el relámpago que relumbra antes de haber podido pestañear.

Por lo tanto, los que no son totalmente conscientes de la desventaja de servirse de las armas no pueden ser totalmente conscientes de las ventajas de utilizarlas.

Los que utilizan los medios militares con pericia no activan a sus tropas dos veces, ni proporcionan alimentos en tres ocasiones, con un mismo objetivo.

Esto quiere decir que no se debe movilizar al pueblo más de una vez por campaña, y que inmediatamente después de alcanzar la victoria no se debe regresar al propio país para hacer una segunda movilización. Al principio esto significa proporcionar alimentos (para las propias tropas), pero después se quitan los alimentos al enemigo.

Si tomas los suministros de armas de tu propio país, pero quitas los alimentos al enemigo, puedes estar bien abastecido de armamento y de provisiones.

Cuando un país se empobrece a causa de las operaciones militares, se debe al transporte de provisiones desde un lugar distante. Si las transportas desde un lugar distante, el pueblo se empobrecerá.
 
Los que habitan cerca de donde está el ejército pueden vender sus cosechas a precios elevados, pero se acaba de este modo el bienestar de la mayoría de la población.

Cuando se transportan las provisiones muy lejos, la gente se arruina a causa del alto costo. En los mercados cercanos al ejército, los precios de las mercancías e aumentan. Por lo tanto, las largas campañas militares constituyen una lacra para el país.

Cuando se agotan los recursos, los impuestos se recaudan bajo presión. Cuando el poder y los recursos se han agotado, se arruina el propio país. Se priva al pueblo de gran parte de su presupuesto, mientras que los gastos del gobierno para armamentos se elevan.

Los habitantes constituyen la base de un país, los alimentos son la felicidad del pueblo. El príncipe debe respetar este hecho y ser sobrio y austero en sus gastos públicos.

En consecuencia, un general inteligente lucha por desproveer al enemigo de sus alimentos. Cada porción de alimentos tomados al enemigo equivale a veinte que te suministras a ti mismo.

Así pues, lo que arrasa al enemigo es la imprudencia, y la motivación de los tuyos en asumir los beneficios de los adversarios.

Cuando recompenses a tus hombres con los beneficios que ostentaban los adversarios los harás luchar por propia iniciativa, y así podrás tomar el poder y la influencia que tenía el enemigo. Es por esto par lo que se dice que donde hay grandes recompensas hay hombres valientes.

Por consiguiente, en una batalla de carros, recompensa primero al que tome al menos diez carros.

Si recompensas a todo el mundo, no habrá suficiente para todos, así pues, ofrece una recompensa a un soldado para animar a todos los demás. Cambia sus colores (de los soldados enemigos hechos prisioneros), utilízalos mezclados con los tuyos. Trata bien a los soldados y préstales atención. Los soldados prisioneros deben ser bien tratados, para conseguir que en el futuro luchen para ti. A esto se llama vencer al adversario e incrementar por añadidura tus propias fuerzas.

Si utilizas al enemigo para derrotar al enemigo, serás poderoso en cualquier lugar a donde vayas.

Así pues, lo más importante en una operación militar es la victoria y no la persistencia. Esta última no es beneficiosa. Un ejército es como el fuego: si no lo apagas, se consumirá por sí mismo.

Por lo tanto, sabemos que el que está a la cabeza del ejército está a cargo de las vidas de los habitantes y de la seguridad de la nación.
 
