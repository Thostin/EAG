\chapter{Sobre los nueve cambios}

Por lo general, las operaciones militares están bajo el del gobernante civil para dirigir al ejército.

El General no debe levantar su campamento en un terreno difícil. Deja que se establezcan relaciones diplomáticas en las fronteras. No permanezcas en un territorio árido ni aislado.

Cuando te halles en un terreno cerrado, prepara alguna estrategia y muévete. Cuando te halles en un terreno mortal, luc ha.

Terreno cerrado significa que existen lugares escarpados que te rodean por todas partes, de manera que el enemigo tiene movilidad, que puede llegar e irse con libertad, pero a ti te es difícil salir y volver.

Cada ruta debe ser estudiada para que sea la mejor. Hay rutas que no debes usar, ejércitos que no han de ser atacados, ciudades que no deben ser rodeadas, terrenos sobre los que no se debe combatir, y órdenes de gobernantes civiles que no deben ser obedecidas.

En consecuencia, los generales que conocen las variables posibles para aprovecharse del terreno saben cómo manejar las fuerzas armadas. Si los generales no saben cómo adaptarse de manera ventajosa, aunque conozcan la condición del terreno, no pueden aprovecharse de él.

Si están al mando de ejércitos, pero ignoran las artes de la total adaptabilidad, aunque conozcan el objetivo a lograr, no pueden hacer que los soldados luchen por él.

Si eres capaz de ajustar la campaña cambiar conforme al ímpetu de las fuerzas, entonces la ventaja no cambia, y los únicos que son perjudicados son los enemigos. Por esta razón, no existe una estructura permanente. Si puedes comprender totalmente este principio, puedes hacer que los soldados actúen en la mejor forma posible.

Por lo tanto, las consideraciones de la persona inteligente siempre incluyen el analizar objetivamente el beneficio y el daño. Cuando considera el beneficio, su acción se expande; cuando considera el daño, sus problemas pueden resolverse.

El beneficio y el daño son interdependientes, y los sabios los tienen en cuenta.

Por ello, lo que retiene a los adversarios es el daño, lo que les mantiene ocupados es la acción, y lo que les motiva es el beneficio.

Cansa a los enemigos manteniéndolos ocupados y no dejándoles respirar. Pero antes de lograrlo, tienes que realizar previamente tu propia labor. Esa labor consiste en desarrollar un ejército fuerte, un pueblo próspero, una sociedad armoniosa y una manera ordenada de vivir.
 
Así pues, la norma general de las operaciones militares consiste en no contar con que el enemigo no acuda, sino confiar en tener los medios de enfrentarte a él; no contar con que el adversario no ataque, sino confiar en poseer lo que no puede ser atacado.

Si puedes recordar siempre el peligro cuando estás a salvo y el caos en tiempos de orden, permanece atento al peligro y al caos mientras no tengan todavía forma, y evítalos antes de que se presenten; ésta es la mejor estrategia de todas.

Por esto, existen cinco rasgos que son peligrosos en los generales. Los que están dispuestos a morir, pueden perder la vida; los que quieren preservar la vida, pueden ser hechos prisioneros; los que son dados a los apasionamientos irracionales, pueden ser ridiculizados; los que son muy puritanos, pueden ser deshonrados; los que son compasivos, pueden ser turbados.

Si te presentas en un lugar que con toda seguridad los enemigos se precipitarán a defender, las personas compasivas se apresurarán invariablemente a rescatar a sus habitantes, causándose a sí mismos problemas y cansancio.

Estos son cinco rasgos que constituyen defectos en los generales y que son desastrosos para las operaciones militares.

Los buenos generales son de otra manera: se comprometen hasta la muerte, pero no se aferran a la esperanza de sobrevivir; actúan de acuerdo con los acontecimientos, en forma racional y realista, sin dejarse llevar por las emociones ni estar sujetos a quedar confundidos. Cuando ven una buena oportunidad, son como tigres, en caso contrario cierran sus puertas. Su acción y su no acción son cuestiones de estrategia, y no pueden ser complacidos ni enfadados.
 
