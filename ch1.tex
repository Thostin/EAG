\chapter{Sobre la evaluación}

Sun Tzu dice: la guerra es de vital importancia para el Estado; es el dominio de la vida o de la muerte, el camino hacia la supervivencia o la pérdida del Imperio: es forzoso manejarla bien. No reflexionar seriamente sobre todo lo que le concierne es dar prueba de una culpable indiferencia en lo que respecta a la conservación o pérdida de lo que nos es más querido; y ello no debe ocurrir entre nosotros.

Hay que valorarla en términos de cinco factores fundamentales, y hacer comparaciones entre diversas condiciones de los bandos rivales, con vistas a determinar el resultado de la guerra.

El primero de estos factores es la doctrina; el segundo, el tiempo; el tercero, el terreno; el cuarto, el mando; y el quinto, la disciplina.

La doctrina significa aquello que hace que el pueblo esté en armonía con su gobernante, de modo que le siga donde sea, sin temer por sus vidas ni a correr cualquier peligro.

El tiempo significa el Ying y el Yang, la noche y el día, el frío y el calor, días despejados o lluviosos, y el cambio de las estaciones.

El terreno implica las distancias, y hace referencia a dónde es fácil o difícil desplazarse, y si es campo abierto o lugares estrechos, y esto influencia las posibilidades de supervivencia.

El mando ha de tener como cualidades: sabiduría, sinceridad, benevolencia, coraje y disciplina.

Por último, la disciplina ha de ser comprendida como la organización del ejército, las graduaciones y rangos entre los oficiales, la regulación de las rutas de suministros, y la provisión de material militar al ejército.

Estos cinco factores fundamentales han de ser conocidos por cada general. Aquel que los domina, vence; aquel que no, sale derrotado. Por lo tanto, al trazar los planes, han de compararse los siguiente siet e factores, valorando cada uno con el mayor cuidado:

¿Qué dirigente es más sabio y capaz?

¿Qué comandante posee el mayor talento?
 
¿Qué ejército obtiene ventajas de la naturaleza y el terreno?

¿En qué ejército se observan mejor las regulaciones y las instrucciones?

¿Qué tropas son más fuertes?

¿Qué ejército tiene oficiales y tropas mejor entrenadas?

¿Qué ejército administra recompensas y castigos de forma más justa?

Mediante el estudio de estos siete factores, seré capaz de adivinar cual de los dos bandos saldrá victorioso y cual será derrotado.

El general que siga mi consejo, es seguro que vencerá. Ese general ha de ser mantenido al mando. Aquel que ignore mi consejo, ciertamente será derrotado. Ese debe ser destituido.

Tras prestar atención a mi consejo y planes, el general debe crear una situación que contribuya a su cumplimiento. Por situación quiero decir que debe tomar en consideración la situación del campo, y actuar de acuerdo con lo que le es ventajoso.

El arte de la guerra se basa en el engaño. Por lo tanto, cuando es capaz de atacar, ha de aparentar incapacidad; cuando las tropas se mueven, aparentar inactividad. Si está cerca del enemigo, ha de hacerle creer que está lejos; si está lejos, aparentar que se está cerca. Poner cebos para atraer al enemigo.

Golpear al enemigo cuando está desordenado. Prepararse contra él cuando está seguro en todas partes. Evitarle durante un tiempo cuando es más fuerte. Si tu oponente tiene un temperamento colérico, intenta irritarle. Si es arrogante, trata de fomentar su egoísmo.

Si las tropas enemigas se hallan bien preparadas tras una reorganización, intenta desordenarlas. Si están unidas, siembra la disensión entre sus filas. Ataca al enemigo cuando no está preparado, y aparece cuando no te espera. Estas son las claves de la victoria para el estratega.

Ahora, si las estimaciones realizadas antes de la batalla indican victoria, es porque los cálculos cuidadosamente realizados muestran que tus condiciones son más favorables que las condiciones del enemigo; si indican derrota, es porque muestran que las condiciones favorables para la batalla son menores. Con una evaluación cuidadosa, uno puede vencer; sin ella, no puede. Mucha menos oportunidad de victoria tendrá aquel que no realiza cálculos en absoluto.

Gracias a este método, se puede examinar la situación, y el resultado aparece claramente. 

