\chapter{Sobre las nueve clases de terreno}

Conforme a las leyes de las operaciones militares, existen nueve clases de terreno. Si intereses locales luchan entre sí en su propio territorio, a éste se le llama terreno de dispersión.

Cuando los soldados están apegados a su casa y combaten cerca de su hogar, pueden ser dispersados con facilidad.

Cuando penetras en un territorio ajeno, pero no lo haces en profundidad, a éste se le llama territorio ligero.

Esto significa que los soldados pueden regresar fácilmente.

El territorio que puede resultarte ventajoso si lo tomas, y ventajoso al enemigo si es él quien lo conquista, se llama terreno clave.

Un terreno de lucha inevitable es cualquier enclave defensivo o paso estratégico.

Un territorio igualmente accesible para ti y para los demás se llama terreno de comunicación.

El territorio que está rodeado por tres territorios rivales y es el primero en proporcionar libre acceso a él a todo el mundo se llama terreno de intersección.

El terreno de intersección es aquel en el que convergen las principales vías de comunicación uniéndolas entre sí: sé el primero en ocuparlo, y la gente tendrá que ponerse de tu lado. Si lo obtienes, te encuentras seguro; si lo pierdes, corres peligro.

Cuando penetras en profundidad en un territorio ajeno, y dejas detrás muchas ciudades y pueblos, a este terreno se le llama difícil.

Es un terreno del que es difícil regresar.

Cuando atraviesas montañas boscosas, desfiladeros abruptos u otros accidentes difíciles de atravesar, a esto se le llama terreno desfavorable.

Cuando el acceso es estrecho y la salida es tortuosa, de manera que una pequeña unidad enemiga puede atacarte, aunque tus tropas sean más numerosas, a éste se le llama terreno cercado.

Si eres capaz de una gran adaptación, puedes atravesar este territorio.

Si sólo puedes sobrevivir en un territorio luchando con rapidez, y si es fácil morir si no lo haces, a éste se le llama terreno mortal.

Las tropas que se encuentran en un terreno mortal están en la misma situación que si se encontraran en una barca que se hunde o en una casa ardiendo.

Así pues, no combatas en un terreno de dispersión, no te detengas en un terreno ligero, no ataques en un terreno clave (ocupado por el enemigo), no dejes que tus tropas sean divididas en un terreno de comunicación. En terrenos de intersección, establece comunicaciones; en terrenos difíciles, entra aprovisionado; en terrenos desfavorables, continúa marchando; en terrenos cercados, haz planes; en terrenos mortales, lucha.

En un terreno de dispersión, los soldados pueden huir. Un terreno ligero es cuando los soldados han penetrado en territorio enemigo, pero todavía no tienen las espaldas cubiertas: por eso, sus mentes no están realmente concentradas y no están listos para la batalla. No es ventajoso atacar al enemigo en un terreno clave; lo que es ventajoso es llegar el primero a él. No debe permitirse que quede aislado el terreno de comunicación, para poder servirse de las rutas de suministros. En terrenos de intersección, estarás a salvo si estableces alianzas; si las pierdes, te encontrarás en peligro. En terrenos difíciles, entrar aprovisionado significa reunir todo lo necesario para estar allí mucho tiempo. En terrenos desfavorables, ya que no puedes atrincherarte en ello, debes apresurarte a salir. En terrenos cercados, introduce tácticas sorpresivas.

Si las tropas caen en un terreno mortal, todo el mundo luchará de manera espontánea. Por esto se dice: ``Sitúa a las tropas en un terreno mortal y sobrevivirán''.

Los que eran antes considerados como expertos en el arte de la guerra eran capaces de hacer que el enemigo perdiera contacto entre su vanguardia y su retaguardia, la confianza entre las grandes y las pequeñas unidades, el interés recíproco para el bienestar de los diferentes rangos, el apoyo mutuo entre gobernantes y gobernados, el alistamiento de soldados y la coherencia de sus ejércitos. Estos expertos entraban en cción cuando les era ventajoso, y se retenían en caso contrario.

Introducían cambios para confundir al enemigo, atacándolos aquí y allá, aterrorizándolos y sembrando en ellos la confusión, de tal manera que no les daban tiempo para hacer planes.

Se podría preguntar cómo enfrentarse a fuerzas enemigas numerosas y bien organizadas que se dirigen hacia ti. La respuesta es quitarles en primer lugar algo que aprecien, y después te escucharán.

La rapidez de acción es el factor esencial de la condición de la fuerza militar, aprovechándose de los errores de los adversarios, desplazándose por caminos que no esperan y atacando cuando no están en guardia.

Esto significa que para aprovecharse de la falta de preparación, de visión y de cautela de los adversarios, es necesario actuar con rapidez, y que si dudas, esos errores no te servirán de nada.

En una invasión, por regla general, cuanto más se adentran los invasores en el territorio ajeno, más fuertes se hacen, hasta el punto de que el gobierno nativo no puede ya expulsarlos.

Escoge campos fértiles, y las tropas tendrán suficiente para comer. Cuida de su salud y evita el cansancio, consolida su energía, aumenta su fuerza. Que los movimientos de tus tropas y la preparación de tus planes sean insondables.

Consolida la energía más entusiasta de tus tropas, ahorra las fuerzas sobrantes, mantén en secreto tus formaciones y tus planes, permaneciendo insondable para los enemigos, y espera a que se produzca un punto vulnerable para avanzar.

Sitúa a tus tropas en un punto que no tenga salida, de manera que tengan que morir antes de poder escapar. Porque, ¿ante la posibilidad de la muerte, ¿qué no estarán dispuestas a hacer? Los guerreros dan entonces lo mejor de sus fuerzas. Cuando se hallan ante un grave peligro, pierden el miedo. Cuando no hay ningún sitio a donde ir, permanecen firmes; cuando están totalmente implicados en un terreno, se aferran a él. Si no tienen otra opción, lucharán hasta el final.

Por esta razón, los soldados están vigilantes sin tener que ser estimulados, se alistan sin tener que ser llamados a filas, son amistosos sin necesidad de promesas, y se puede confiar en ellos sin necesidad de órdenes.

Esto significa que cuando los combatientes se encuentran en peligro de muerte, sea cual sea su rango, todos tienen el mismo objetivo, y, por lo tanto, están alerta sin necesidad de ser estimulados, tienen buena voluntad de manera espontánea y sin necesidad de recibir órdenes, y puede confiarse de manera natural en ellos sin promesas ni necesidad de jerarquía.

Prohibe los augurios para evitar las dudas, y los soldados nunca te abandonarán. Si tus soldados no tienen riquezas, no es porque las desdeñen. Si no tienen más longevidad, no es porque no quieran vivir más tiempo. El día en que se da la orden de marcha, los soldados lloran.

Así pues, una operación militar preparada con pericia debe ser como una serpiente veloz que contraataca con su cola cuando alguien le ataca por la cabeza, contraataca con la cabeza cuando alguien le ataca por la cola y contraataca con cabeza y cola, cuando alguien le ataca por el medio.

Esta imagen representa el método de una línea de batalla que responde velozmente cuando es atacada. Un manual de ocho formaciones clásicas de batalla dice: ``Haz del frente la retaguardia, haz de la retaguardia el frente, con cuatro cabezas y ocho colas. Haz que la cabeza esté en todas partes, y cuando el enemigo arremeta por el centro, cabeza y cola acudirán al rescate''.

Puede preguntarse la cuestión de si es posible hacer que una fuerza militar sea como una serpiente rápida. La respuesta es afirmativa. Incluso las personas que se tienen antipatía, encontrándose en el mismo barco, se ayudarán entre sí en caso de peligro de zozobrar.

Es la fuerza de la situación la que hace que esto suceda.

Por esto, no basta con depositar la confianza en caballos atados y ruedas fijadas.

Se atan los caballos para formar una línea de combate estable, y se fijan las ruedas para hacer que los carros no se puedan mover. Pero, aun así, esto no es suficientemente seguro ni se puede confiar en ello. Es necesario permitir que haya variantes a los cambios que se hacen, poniendo a los soldados en situaciones mortales, de manera que combatan de forma espontánea y se ayuden unos a otros, codo con codo: éste es el camino de la seguridad y de la obtención de una victoria cierta.

La mejor organización es hacer que se exprese el valor y mantenerlo constante. Tener éxito tanto con tropas débiles como con tropas aguerridas se basa en la configuración de las circunstancias.

Si obtienes la ventaja del terreno, puedes vencer a los adversarios, incluso con tropas ligeras y débiles; ¿cuánto más te sería posible si tienes tropas poderosas y aguerridas? Lo que hace posible la victoria a ambas clases de tropas es las circunstancias del terreno.

Por lo tanto, los expertos en operaciones militares logran la cooperación de la tropa, de tal manera que dirigir un grupo es como dirigir a un solo individuo que no tiene más que una sola opción.

Corresponde al general ser tranquilo, reservado, justo y metódico.

Sus planes son tranquilos y absolutamente secretos para que nadie pueda descubrirlos. Su mando es justo y metódico, así que nadie se atreve a tomarlo a la ligera.

Puede mantener a sus soldados sin información y en completa ignorancia de sus planes.

Cambia sus acciones y revisa sus planes, de manera que nadie pueda reconocerlos. Cambia de lugar su emplazamiento y se desplaza por caminos sinuosos, de manera que nadie pueda anticiparse.

Puedes ganar cuando nadie puede entender en ningún momento cuáles son tus intenciones.

Dice un Gran Hombre: ``El principal engaño que se valora en las operaciones militares no se dirige sólo a los enemigos, sino que empieza por las propias tropas, para hacer que le sigan a uno sin saber adónde van''. Cuando un general fija una meta a sus tropas, es como el que sube a un lugar elevado y después retira la escalera. Cuando un general se adentra muy en el interior del territorio enemigo, está poniendo a prueba todo su potencial.

Ha hecho quemar las naves a sus tropas y destruir sus casas; así las conduce como un rebaño y todos ignoran hacia dónde se encaminan.

Incumbe a los generales reunir a los ejércitos y ponerlos en situaciones peligrosas. También han de examinar las adaptaciones a los diferentes terrenos, las ventajas de concentrarse o dispersarse, y las pautas de los sentimientos y situaciones humanas.

Cuando se habla de ventajas y de desventajas de la concentración y de la dispersión, quiere decir que las pautas de los comportamientos humanos cambian según los diferentes tipos de terreno.

En general, la pauta general de los invasores es unirse cuando están en el corazón del territorio enemigo, pero tienden a dispersarse cuando están en las franjas fronterizas. Cuando dejas tu territorio y atraviesas la frontera en una operación militar, te hayas en un terreno aislado.

Cuando es accesible desde todos los puntos, es un terreno de comunicación.

Cuando te adentras en profundidad, estás en un terreno difícil. Cuando penetras poco, estás en un terreno ligero.

Cuando a tus espaldas se hallen espesuras infranqueables y delante pasajes estrechos, estás en un terreno cercado.

Cuando no haya ningún sitio a donde ir, se trata de un terreno mortal.

Así pues, en un terreno de dispersión, yo unificaría las mentes de los soldados. En un terreno ligero, las mantendría en contacto. En un terreno clave, les haría apresurarse para tomarlo. En un terreno de intersección, prestaría atención a la defensa. En un terreno de comunicación, establecería sólidas alianzas. En un terreno difícil, aseguraría suministros continuados. En un terreno desfavorable, urgiría a mis tropas a salir rápidamente de él. En un terreno cercado, cerraría las entradas. En un terreno mortal, indicaría a mis tropas que no existe ninguna posibilidad de sobrevivir.

Por esto, la psicología de los soldados consiste en resistir cuando se ven rodeados, luchar cuando no se puede evitar, y obedecer en casos extremos.

Hasta que los soldados no se ven rodeados, no tienen la determinación de resistir al enemigo hasta alcanzar la victoria. Cuando están desesperados, presentan una defensa unificada.

Por ello, los que ignoran los planes enemigos no pueden preparar alianzas.

Los que ignoran las circunstancias del terreno no pueden hacer maniobrar a sus fuerzas. Los que no utilizan guías locales no pueden aprovecharse del terreno. Los militares de un gobierno eficaz deben conocer todos estos factores.

Cuando el ejército de un gobierno eficaz ataca a un gran territorio, el pueblo no se puede unir. Cuando su poder sobrepasar a los adversarios, es imposible hacer alianzas.

Si puedes averiguar los planes de tus adversarios, aprovéchate del terreno y haz maniobrar al enemigo de manera que se encuentre indefenso; en este caso, ni siquiera un gran territorio puede reunir suficientes tropas para detenerte.

Por lo tanto, si no luchas por obtener alianzas, ni aumentas el poder de ningún país, pero extiendes tu influencia personal amenazando a los adversarios, todo ello hace que el país y las ciudades enemigas sean vulnerables.

Otorga recompensas que no estén reguladas y da órdenes desacostumbradas.

Considera la ventaja de otorgar recompensas que no tengan precedentes, observa cómo el enemigo hace promesas sin tener en cuenta los códigos establecidos.

Maneja las tropas como si fueran una sola persona. Empléalas en tareas reales, pero no les hables. Motívalas con recompensas, pero no les comentes los perjuicios posibles.

Emplea a tus soldados sólo en combatir, sin comunicarles tu estrategia. Déjales conocer los beneficios que les esperan, pero no les hables de los daños potenciales. Si la verdad se filtra, tu estrategia puede hundirse. Si los soldados empiezan a preocuparse, se volverán vacilantes y temerosos.

Colócalos en una situación de posible exterminio, y entonces lucharán para vivir. Ponles en peligro de muerte, y entonces sobrevivirán. Cuando las tropas afrontan peligros, son capaces de luchar para obtener la victoria.

Así pues, la tarea de una operación militar es fingir acomodarse a las intenciones del enemigo. Si te concentras totalmente en éste, puedes matar a su general, aunque estés a kilómetros de distancia. A esto se llama cumplir el objetivo con pericia.

Al principio te acomodas a sus intenciones, después matas a sus generales: ésta es la pericia en el cumplimiento del objetivo.

Así, el día en que se declara la guerra, se cierran las fronteras, se rompen los salvoconductos y se impide el paso de emisarios.

Los asuntos se deciden rigurosamente desde que se comienza a planificar y establecer la estrategia desde la casa o cuartel general.

El rigor en los cuarteles generales en la fase de planificación se refiere al mantenimiento del secreto.

Cuando el enemigo ofrece oportunidades, aprovéchalas inmediatamente.

Entérate primero de lo que pretende, y después anticípate a él. Mantén la disciplina y adáptate al enemigo, para determinar el resultado de la guerra. Así, al principio eres como una doncella y el enemigo abre sus puertas; entonces, tú eres como una liebre suelta, y el enemigo no podrá expulsarte.

