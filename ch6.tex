\chapter{Sobre lo lleno y lo vacío}

Los que anticipan, se preparan y llegan primero al campo de batalla y esperan al adversario están en posición descansada; los que llegan los últimos al campo de batalla, los que improvisan y entablan la lucha quedan agotados.

Los buenos guerreros hacen que los adversarios vengan a ellos, y de ningún modo se dejan atraer fuera de su fortaleza.

Si haces que los adversarios vengan a ti para combatir, su fuerza estará siempre vacía. Si no sales a combatir, tu fuerza estará siempre llena. Este es el arte de vaciar a los demás y de llenarte a ti mismo.

Lo que impulsa a los adversarios a venir hacia ti por propia decisión es la perspectiva de ganar. Lo que desanima a los adversarios de ir hacia ti es la probabilidad de sufrir daños.

Cuando los adversarios están en posición favorable, debes cansarlos. Cuando están bien alimentados, cortar los suministros. Cuando están descansando, hacer que se pongan en movimiento.

Ataca inesperadamente, haciendo que los adversarios se agoten corriendo para salvar sus vidas. Interrumpe sus provisiones, arrasa sus campos y corta sus vías de aprovisionamiento. Aparece en lugares críticos y ataca donde menos se lo esperen, haciendo que tengan que acudir al rescate.

Aparece donde no puedan ir, se dirige hacia donde menos se lo esperen. Para desplazarte cientos de kilómetros sin cansancio, atraviesa tierras despobladas.

Atacar un espacio abierto no significa sólo un espacio en el que el enemigo no tiene defensa. Mientras su defensa no sea estricta ---el lugar no esté bien guardado---, los enemigos se desperdigarán ante ti, como si estuvieras atravesando un territorio despoblado.

Para tomar infaliblemente lo que atacas, ataca donde no haya defensa. Para mantener una defensa infaliblemente segura, defiende donde no haya ataque.

Así, en el caso de los que son expertos en el ataque, sus enemigos no saben por dónde atacar.

Cuando se cumplen las instrucciones, las personas son sinceramente leales y comprometidas, los planes y preparativos para la defensa implantados con firmeza, siendo tan sutil y reservado que no se revelan las estrategias de ninguna forma, y los adversarios se sienten inseguros, y su inteligencia no les sirve para nada.

Sé extremadamente sutil, discreto, hasta el punto de no tener forma. Sé completamente misterioso y confidencial, hasta el punto de ser silencioso. De esta manera podrás dirigir el destino de tus adversarios.

Para avanzar sin encontrar resistencia, arremete por sus puntos débiles. Para retirarte de manera esquiva, sé más rápido que ellos.

Las situaciones militares se basan en la velocidad: llega como el viento, muévete como el relámpago, y los adversarios no podrán vencerte.

Por lo tanto, cuando quieras entrar en batalla, incluso si el adversario está atrincherado en una posición defensiva, no podrá evitar luchar si atacas en el lugar en el que debe acudir irremediablemente al rescate.

Cuando no quieras entrar en batalla, incluso si trazas una línea en el terreno que quieres conservar, el adversario no puede combatir contigo porque le das una falsa pista.

Esto significa que cuando los adversarios llegan para atacarte, no luchas con ellos, sino que estableces un cambio estratégico para confundirlos y llenarlos de incertidumbre.

Por consiguiente, cuando induces a otros a efectuar una formación, mientras que tú mismo permaneces sin forma, estás concentrado, mientras que tu adversario está dividido.

Haz que los adversarios vean como extraordinario lo que es ordinario para ti; haz que vean como ordinario lo que es extraordinario para ti. Esto es inducir al enemigo a efectuar una formación. Una vez vista la formación del adversario, concentras tus tropas contra él. Como tú formación no está a la vista, el adversario dividirá seguramente sus fuerzas.

Cuando estás concentrado formando una sola fuerza, mientras que el enemigo está dividido en diez, estás atacando a una concentración de uno contra diez, así que tus fuerzas superan a las suyas.

Si puedes atacar a unos pocos soldados con muchos, diezmarás el número de tus adversarios.

Cuando estás fuertemente atrincherado, te has hecho fuerte tras buenas barricadas, y no dejas filtrar ninguna información sobre tus fuerzas, sal afuera sin formación precisa, ataca y conquista de manera incontenible.

No han de conocer dónde piensas librar la batalla, porque cuando no se conoce, el enemigo destaca muchos puestos de vigilancia, y en el momento en el que se establecen numerosos puestos sólo tienes que combatir contra pequeñas unidades.

Así pues, cuando su vanguardia está preparada, su retaguardia es defectuosa, y cuando su retaguardia está preparada, su vanguardia presenta puntos débiles.

Las preparaciones de su ala derecha significarán carencia en su ala izquierda. La preparación por todas partes significará ser vulnerable por todas partes.

Esto significa que cuando las tropas están de guardia en muchos lugares, están forzosamente desperdigadas en pequeñas unidades.

Cuando se dispone de pocos soldados se está a la defensiva contra el adversario el que dispone de muchos hace que el enemigo tenga que defenderse.

Cuantas más defensas induces a adoptar a tu enemigo, más debilitado quedará.

Así, si conoces el lugar y la fecha de la batalla, puedes acudir a ella, aunque estés a mil kilómetros de distancia. Si no conoces el lugar y la fecha de la batalla, entonces tu flanco izquierdo no puede salvar al derecho, tu vanguardia no puede salvar a tu retaguardia, y tu retaguardia no puede salvar a tu vanguardia, ni siquiera en un territorio de unas pocas docenas de kilómetros.

Si tienes muchas más tropas que los demás, ¿cómo puede ayudarte este factor para obtener la victoria?

Si no conoces el lugar y la fecha de la batalla, aunque tus tropas sean más numerosas que las de ellos, ¿cómo puedes saber si vas a ganar o a perder?

Así pues, se dice que la victoria puede ser creada.

Si haces que los adversarios no sepan el lugar y la fecha de la batalla, siempre puedes vencer.

Incluso si los enemigos son numerosos, puede hacerse que no entren en combate.

Por tanto, haz tu valoración sobre ellos para averiguar sus planes, y determinar qué estrategia puede tener éxito y cuál no. Incítalos a la acción para descubrir cuál es el esquema general de sus movimientos y descansa.

Haz algo por o en contra de ellos para su atención, de manera que puedas de ellos para atraer descubrir sus hábitos de comportamiento de ataque y de defensa.

Indúcelos a adoptar formaciones específicas, para conocer sus puntos flacos.

Esto significa utilizar muchos métodos para confundir y perturbar al enemigo con el objetivo de observar sus formas de respuesta hacia ti; después de haberlas observado, actúas en consecuencia, de manera que puedes saber qué clase de situaciones significan vida y cuáles significan muerte.

Pruébalos para averiguar sus puntos fuertes y sus puntos débiles. Por lo tanto, el punto final de la formación de un ejército es llegar a la no forma. Cuando no tienes forma, los informadores no pueden descubrir nada, ya que la información no puede crear una estrategia.

Una vez que no tienes forma perceptible, no dejas huellas que puedan ser seguidas, los informadores no encuentran ninguna grieta por donde mirar y los que están a cargo de la planificación no pueden establecer ningún plan realizable.

La victoria sobre multitudes mediante formaciones precisas debe ser desconocida para las multitudes. Todo el mundo conoce la forma mediante la que resultó vencedor, pero nadie conoce la forma mediante la que aseguró la victoria.

En consecuencia, la victoria en la guerra no es repetitiva, sino que adapta su forma continuamente.

Determinar los cambios apropiados, significa no repetir las estrategias previas para obtener la victoria. Para lograrla, puedo adaptarme desde el principio a cualquier formación que los adversarios puedan adoptar.

Las formaciones son como el agua: la naturaleza del agua es evitar lo alto e ir hacia abajo; la naturaleza de los ejércitos es evitar lo lleno y atacar lo vacío; el flujo del agua está determinado par la tierra; la victoria viene determinada por el adversario.

Así pues, un ejército no tiene formación constante, lo mismo que el agua no tiene forma constante: se llama genio a la capacidad de obtener la victoria cambiando y adaptándose según el enemigo.
