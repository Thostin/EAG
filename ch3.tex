\chapter{Sobre las proposiciones de la victoria y la derrota}

Como regla general, es mejor conservar a un enemigo intacto que destruirlo. Capturar a sus soldados para conquistarlos y dominas a sus jefes.

Un General decía: ``Practica las artes marciales, calcula la fuerza de tus adversarios, haz que pierdan su ánimo y dirección, de manera que, aunque el ejército enemigo esté intacto sea inservible: esto es ganar sin violencia. Si destruyes al ejército enemigo y matas a sus generales, asaltas sus defensas disparando, reúnes a una muchedumbre y usurpas un territorio, todo esto es ganar por la fuerza''.

Por esto, los que ganan todas las batallas no son realmente profesionales; los que consiguen que se rindan impotentes los ejércitos ajenos sin luchar son los mejores maestros del Arte de la Guerra.

Los guerreros superiores atacan mientras los enemigos están proyectando sus planes. Luego deshacen sus alianzas.

Por eso, un gran emperador decía: ``El que lucha por la victoria frente a espadas desnudas no es un buen general''. La peor táctica es atacar a una ciudad. Asediar, acorralar a una ciudad sólo se lleva a cabo co mo último recurso.

Emplea no menos de tres meses en preparar tus artefactos y otros tres para coordinar los recursos para tu asedio. Nunca se debe atacar por cólera y con prisas. Es aconsejable tomarse tiempo en la planificación y coordinación del plan.

Por lo tanto, un verdadero maestro de las artes marciales vence a otras fuerzas enemigas sin batalla, conquista otras ciudades sin asediarlas y destruye a otros ejércitos sin emplear mucho tiempo.

Un maestro experto en las artes marciales deshace los planes de los enemigos, estropea sus relaciones y alianzas, le corta los suministros o bloquea su camino, venciendo mediante estas tácticas sin necesidad de luchar.

Es imprescindible luchar contra todas las facciones enemigas para obtener una victoria completa, de manera que su ejército no quede acuartelado y el beneficio sea total. Esta es la ley del asedio estratégico.

La victoria completa se produce cuando el ejército no lucha, la ciudad no es asediada, la destrucción no se prolonga durante mucho tiempo, y en cada caso el enemigo es vencido por el empleo de la estrategia.

Así pues, la regla de la utilización de la fuerza es la siguiente: si tus fuerzas son diez veces superiores a las del adversario, rodéalo; si son cinco veces superiores, atácalo; si son dos veces superiores, divídelo.

Si tus fuerzas son iguales en número, lucha si te es posible. Si tus fuerzas son inferiores, manténte continuamente en guardia, pues el más pequeño fallo te acarrearía las peores consecuencias. Trata de mantenerte al abrigo y evita en lo posible un enfrentamiento abierto con él; la prudencia y la firmeza de un pequeño número de personas pueden llegar a cansar y a dominar incluso a numerosos ejércitos.
 
Este consejo se aplica en los casos en que todos los factores son equivalentes. Si tus fuerzas están en orden mientras que las suyas están inmersas en el caos, si tú y tus fuerzas están con ánimo y ellos desmoralizados, entonces, aunque sean más numerosos, puedes entrar en batalla. Si tus soldados, tus fuerzas, tu estrategia y tu valor son menores que las de tu adversario, entonces debes retirarte y buscar una salida.

En consecuencia, si el bando más pequeño es obstinado, cae prisionero del bando más grande.

Esto quiere decir que, si un pequeño ejército no hace una valoración adecuada de su poder y se atreve a enemistarse con una gran potencia, por mucho que su defensa sea firme, inevitablemente se convertirá en conquistado. ``Si no puedes ser fuerte, pero tampoco sabes ser débil, serás derrotado''. Los generales son servidores del Pueblo. Cuando su servicio es completo, el Pueblo es fuerte. Cuando su servicio es defectuoso, el Pueblo es débil.

Así pues, existen tres maneras en las que un Príncipe lleva al ejército al desastre. Cuando un Príncipe, ignorando los hechos, ordena avanzar a sus ejércitos o retirarse cuando no deben hacerlo; a esto se le llama inmovilizar al ejército. Cuando un Príncipe ignora los asuntos militares, pero comparte en pie de igualdad el mando del ejército, los soldados acaban confusos. Cuando el Príncipe ignora cómo llevar a cabo las maniobras militares, pero comparte por igual su dirección, los soldados están vacilantes. Una vez que los ejércitos están confusos y vacilantes, empiezan los problemas procedentes de los adversarios. A esto se le llama perder la victoria por trastornar el aspecto militar.

Si intentas utilizar los métodos de un gobierno civil para dirigir una operación militar, la operación será confusa.

Triunfan aquellos que:

Saben cuándo luchar y cuándo no

Saben discernir cuándo utilizar muchas o pocas tropas.

Tienen tropas cuyos rangos superiores e inferiores tienen el mismo objetivo. Se enfrentan con preparativos a enemigos desprevenidos.

Tienen generales competentes y no limitados por sus gobiernos civiles. Estas cinco son las maneras de conocer al futuro vencedor.

Hablar de que el Príncipe sea el que da las órdenes en todo es como el General solicitarle permiso al Príncipe para poder apagar un fuego: para cuando sea autorizado, ya no quedan sino cenizas.

Si conoces a los demás y te conoces a ti mismo, ni en cien batallas correrás peligro; si no conoces a los demás, pero te conoces a ti mismo, perderás una batalla y ganarás otra; si no conoces a los demás ni te conoces a ti mismo, correrás peligro en cada batalla.

