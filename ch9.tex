\chapter{Sobre la distribución de los medios}

Las maniobras militares son el resultado de los planes y las estrategias en la manera más ventajosa para ganar. Determinan la movilidad y efectividad de las tropas.

Si vas a colocar tu ejército en posición de observar al enemigo, atraviesa rápido las montañas y vigílalos desde un valle.

Considera el efecto de la luz y manténte en la posición más elevada del valle. Cuando combatas en una montaña, ataca desde arriba hacia abajo y no al revés.

Combate estando cuesta abajo y nunca cuesta arriba. Evita que el agua divida tus fuerzas, aléjate de las condiciones desfavorables lo antes que te sea posible. No te enfrentes a los enemigos dentro del agua; es conveniente dejar que pasen la mitad de sus tropas y en ese momento dividirlas y atacarlas.

No te sitúes río abajo. No camines en contra de la corriente, ni en contra del viento.

Si acampas en la ribera de un río, tus ejércitos pueden ser sorprendidos de noche, empujados a ahogarse o se les puede colocar veneno en la corriente. Tus barcas no deben ser amarradas corriente abajo, para impedir que el enemigo aproveche la corriente lanzando sus barcas contra ti. Si atraviesas pantanos, hazlo rápidamente. Si te encuentras frente a un ejército en media de un pantano, permanece cerca de sus plantas acuáticas o respaldado por los árboles.

En una llanura, toma posiciones desde las que sea fácil maniobrar, manteniendo las elevaciones del terreno detrás y a tu derecha, estando las partes más bajas delante y las más altos detrás.

Generalmente, un ejército prefiere un terreno elevado y evita un terreno bajo, aprecia la luz y detesta la oscuridad.

Los terrenos elevados son estimulantes, y, por lo tanto, la gente se halla a gusto en ellos, además son convenientes para adquirir la fuerza del ímpetu. Los terrenos bajos son húmedos, lo cual provoca enfermedades y dificulta el combate.

Cuida de la salud física de tus soldados con los mejores recursos disponibles. Cuando no existe la enfermedad en un ejército, se dice que éste es invencible.

Donde haya montículos y terraplenes, sitúate en su lado soleado, manteniéndolos siempre a tu derecha y detrás. 

Colocarse en la mejor parte del terreno es ventajoso para una fuerza militar.

La ventaja en una operación militar consiste en aprovecharse de todos los factores beneficiosos del terreno.

Cuando llueve río arriba y la corriente trae consigo la espuma, si quieres cruzarlo, espera a que escampe.

Siempre que un terreno presente barrancos infranqueables, lugares cerrados, trampas, riesgos, grietas y prisiones naturales, debes abandonarlo rápidamente y no acercarte a él. En lo que a mí concierne, siempre me mantengo alejado de estos accidentes del terreno, de manera que los adversarios estén más cerca que yo de ellos; doy la cara a estos accidentes, de manera que queden a espaldas del enemigo.

Entonces estás en situación ventajosa, y él tiene condiciones desfavorables.

Cuando un ejército se está desplazando, si atraviesa territorios montañosos con muchas corrientes de agua y pozos, o pantanos cubiertos de juncos, o bosques vírgenes llenos de árboles y vegetación, es imprescindible escudriñarlos totalmente y con cuidado, ya que estos lugares ayudan a las emboscadas y a los espías.

Es esencial bajar del caballo y escudriñar el terreno, por si existen tropas escondidas para tenderte una emboscada. También podría ser que hubiera espías al acecho observándote y escuchando tus instrucciones y movimientos.

Cuando el enemigo está cerca, pero permanece en calma, quiere decir que se halla en una posición fuerte. Cuando está lejos, pero intenta provocar hostilidades, quiere que avances. Si, además, su posición es accesible, eso quiere decir que le es favorable.

Si un adversario no conserva la posición que le es favorable por las condiciones del terreno y se sitúa en otro lugar conveniente, debe ser porque existe alguna ventaja táctica para obrar de esta manera.

Si se mueven los árboles, es que el enemigo se está acercando. Si hay obstáculos entre los matorrales, es que has tomado un mal camino.

La idea de poner muchos obstáculos entre la maleza es hacerte pensar que existen tropas emboscadas escondidas en medio de ella.

Si los pájaros alzan el vuelo, hay tropas emboscadas en el lugar. Si los animales están asustados, existen tropas atacantes. Si se elevan columnas de polvo altas y espesas, hay carros que se están acercando; si son bajas y anchas, se acercan soldados a pie. Humaredas esparcidas significan que se está cortando leña. Pequeñas polvaredas que van y vienen indican que hay que levantar el campamento.

Si los emisarios del enemigo pronuncian palabras humildes mientras que éste incrementa sus preparativos de guerra, esto quiere decir que va a avanzar. Cuando se pronuncian palabras altisonantes y se avanza ostentosamente, es señal de que el enemigo se va a retirar.

Si sus emisarios vienen con palabras humildes, envía espías para observar al enemigo y comprobarás que está aumentando sus preparativos de guerra.

Cuando los carros ligeros salen en primer lugar y se sitúan en los flancos, están estableciendo un frente de batalla.

Si los emisarios llegan pidiendo la paz sin firmar un tratado, significa que están tramando algún complot.

Si el enemigo dispone rápidamente a sus carros en filas de combate, es que está esperando refuerzos.

No se precipitarán para un encuentro ordinario si no entienden que les ayudará, o debe haber una fuerza que se halla a distancia y que es esperada en un determinado momento para unir sus tropas y atacarte. Conviene anticipar, prepararse inmediatamente para esta eventualidad.

Si la mitad de sus tropas avanza y la otra mitad retrocede, es que el enemigo piensa atraerte a una trampa.

El enemigo está fingiendo en este caso confusión y desorden para incitarte a que avances. Si los soldados enemigos se apoyan unos en otros, es que están hambrientos.

Si los aguadores beben en primer lugar, es que las tropas están sedientas. Si el enemigo ve una ventaja, pero no la aprovecha, es que está cansado. Si los pájaros se reúnen en el campo enemigo, es que el lugar está vacío. Si hay pájaros sobrevolando una ciudad, el ejército ha huido.

Si se producen llamadas nocturnas, es que los soldados enemigos están atemorizados. Tienen miedo y están inquietos, y por eso se llaman unos a otros.

Si el ejército no tiene disciplina, esto quiere decir que el general no es tomado en serio. Si los estandartes se mueven, es que está sumido en la confusión.

Las señales se utilizan para unificar el grupo; así pues, si se desplaza de acá para allá sin orden ni concierto, significa que sus filas están confusas.

Si sus emisarios muestran irritación, significa que están cansados.

Si matan sus caballos para obtener carne, es que los soldados carecen de alimentos; cuando no tienen marmitas y no vuelven a su campamento, son enemigos completamente desesperados.

Si se producen murmuraciones, faltas de disciplina y los soldados hablan mucho entre sí, quiere decir que se ha perdido la lealtad de la tropa.

Las murmuraciones describen la expresión de los verdaderos sentimientos; las faltas de disciplina indican problemas con los superiores. Cuando el mando ha perdido la lealtad de las tropas, los soldados se hablan con franqueza entre sí sobre los problemas con sus superiores.

Si se otorgan numerosas recompensas, es que el enemigo se halla en un callejón sin salida; cuando se ordenan demasiados castigos, es que el enemigo estádesesperado.

Cuando la fuerza de su ímpetu está agotada, otorgan constantes recompensas para tener contentos a los soldados, para evitar que se rebelen en masa. Cuando los soldados están tan agotados que no pueden cumplir las órdenes, son castigados una y otra vez para restablecer la autoridad.

Ser violento al principio y terminar después temiendo a los propios soldados es el colmo de la ineptitud.

Los emisarios que acuden con actitud conciliatoria indican que el enemigo quiere una tregua.

Si las tropas enemigas se enfrentan a ti con ardor, pero demoran el momento de entrar en combate sin abandonar no obstante el terreno, has de observarlos cuidadosamente.

Están preparando un ataque por sorpresa.

En asuntos militares, no es necesariamente más beneficioso ser superior en fuerzas, sólo evitar actuar con violencia innecesaria; es suficiente con consolidar tu poder, hacer estimaciones sobre el enemigo y conseguir reunir tropas; eso es todo.

El enemigo que actúa aisladamente, que carece de estrategia y que toma a la ligera a sus adversarios, inevitablemente acabará siendo derrotado.

Si tu plan no contiene una estrategia de retirada o posterior al ataque, sino que confías exclusivamente en al fuerza de tus soldados, y tomas a la ligera a tus adversarios sin valorar su condición, con toda seguridad caerás prisionero.

Si se castiga a los soldados antes de haber conseguido que sean leales al mando, no obedecerán, y si no obedecen, serán difíciles de emplear.

Tampoco podrán ser empleados si no se lleva a cabo ningún castigo, incluso después de haber obtenido su lealtad.

Cuando existe un sentimiento subterráneo de aprecio y confianza, y los corazones de los soldados están ya vinculados al mando, si se relaja la disciplina, los soldados se volverán arrogantes y será imposible emplearlos.

Por lo tanto, dirígelos mediante el arte civilizado y unifícalos mediante las a r t e s marciales; esto significa una victoria continua.

Arte civilizado significa humanidad, y artes marciales significan reglamentos. Mándalos con humanidad y benevolencia, unifícalos de manera estricta y firme. Cuando la benevolencia y la firmeza son evidentes, es posible estar seguro de la victoria.

Cuando las órdenes se dan de manera clara, sencilla y consecuente a las tropas, éstas las aceptan. Cuando las órdenes son confusas, contradictorias y cambiantes las tropas no las aceptan o no las entienden.

Cuando las órdenes son razonables, justas, sencillas, claras y consecuentes, existe una satisfacción recíproca entre el líder y el grupo.

